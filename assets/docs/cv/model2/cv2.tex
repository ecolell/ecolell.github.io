	% start of file `template_en.tex'.
% Copyright 2007 Xavier Danaux (xdanaux@gmail.com).
%
% This work may be distributed and/or modified under the
% conditions of the LaTeX Project Public License version 1.3c,
% available at http://www.latex-project.org/lppl/.


\documentclass[11pt,a4paper]{moderncv}

% moderncv themes
%\moderncvtheme[blue]{casual}                 % optional argument are 'blue' (default), 'orange', 'red', 'green', 'grey' and 'roman' (for roman fonts, instead of sans serif fonts)
\moderncvtheme[blue]{classic}                % idem
\usepackage[bottom]{footmisc}

% character encoding

\usepackage[utf8]{inputenc}                   % replace by the encoding you are using
\usepackage{CJK}

% adjust the page margins
\usepackage[scale=0.8]{geometry}
\recomputelengths                             % required when changes are made to page layout lengths

% personal data
\firstname{Eloy Adonis}
\familyname{Colell}
\title{Master en Bioinformática \protect\\y Biología de Sistemas}        % optional, remove the line if not wante
\address{San Nicolás 2310}{B2700LCR, Pergamino}    % optional, remove the line if not wanted
\mobile{+54 9 02477 15536539}                            % optional, remove the line if not wanted
% \phone{phone (optional)}                    % optional, remove the line if not wanted
% \fax{fax (optional)}                        % optional, remove the line if not wanted
\email{eloy.colell.jobs@gmail.com}                      % optional, remove the line if not wanted
\homepage{ecolell.github.io}
\extrainfo{37 años} % optional, remove the line if not wanted
\photo[50pt]{foto/fotoCarnet10.jpg}                % '64pt' is the height the picture must be resized to and 'picture' is the name of the picture file; optional, remove the line if not wanted
% \quote{Some quote (optional)}               % optional, remove the line if not wanted

%\nopagenumbers{}                              % uncomment to suppress automatic page numbering for CVs longer than one page


%----------------------------------------------------------------------------------
%            content
%----------------------------------------------------------------------------------
\begin{document}

\maketitle

\hyphenation{Fa-cul-tad}

\section{Formación Académica}
% \cventry{year--year}{Degree}{Institution}{City}{\textit{Grade}}{Description}  % arguments 3 to 6 are optional
\cventry{2014--2018}{Master en Bioinformática y Biología de Sistemas}{Universidad Nacional del Noroeste de la Provincia de Buenos Aires (UNNOBA)}{Pergamino}{}{}
\cventry{2005--2013}{Licenciado en Sistemas de Información}{Universidad Nacional de Luján (UNLu)}{Luján}{}{}
\cventry{2002--2005}{Analista de Sistemas de Información}{Universidad Nacional de Luján (UNLu)}{Pergamino}{}{}
\cventry{1999--2001}{Bachiller Polimodal en Ciencias Naturales}{Institulo Comercial Rancagua}{Rancagua}{}{}
\cventry{1997--1998}{EGB}{Escuela $N^{o}$ 54}{Rancagua}{}{}
\cventry{1989--1996}{EGB}{Escuela $N^{o}$ 62}{Pergamino}{}{}

% section{Master thesis}
% \cvline{title}{\emph{Title}}
% \cvline{supervisors}{Supervisors}
% \cvline{description}{\small Short thesis abstract}


\section{Publicaciones}
% \cventry{year--year}{Job title}{Author}{}{Issue}{Description}  % arguments 3 to 6 are optional
\cventry{2018}{MISTIC2: comprehensive server to study coevolution in protein families}{Colell, EA et. al.}{}{Nucleic Acids Research}{}


\section{Experiencia Académica}
% \cventry{year--year}{Job title}{Employer}{City}{}{Description}  % arguments 3 to 6 are optional
\cventry{2018}{Desarrollo de una herramienta bioinformática para el
estudio de coevolución en familias de proteínas}{Tésis de master}{}{}{Se crea una plataforma en colaboración con el Instituto Leloir para evaluar distintos algorítmos para estimar coevolución entre dos posiciones proteicas a partir de una familia de proteínas. Ver: \url{https://mistic2.leloir.org.ar}}
\cventry{2013}{Adaptabilidad al contexto en aplicaciones web basadas en Continuations}{Tésis de grado}{}{}{Presenta un modelo para incorporar la adaptabilidad al contexto en aplicaciones web basadas continuations, en donde se logra preservar parte de la privacidad del usuario.}
\cventry{2012}{Obtención de la radiación solar a partir de imágenes MSG}{Curso de especialización en procesamiento de imágenes satelitales realizado en la Universidad de Jaén (España)}{}{}{Se replica un estudio ya realizado sobre el area de Andalucía (España) utilizando imágenes del satélite Meteosat Segunda Generación.}
\cventry{2011}{Seminario de Coaching y Liderazgo}{Dictado por Lic. Fabiola Robin Marquez}{}{}{Se introducen los tipos de liderazgos, la motivación, la influencia, el conflicto/mediación/resolución, la toma de decisiones y el coaching.}
\cventry{2010}{Conceptos Básicos de Geoestadística}{Seminario de Actualización ll}{}{}{Libro introducción a los conceptos básicos de la Geoestadística.}
\cventry{2009}{Fundamentos del testing y técnicas de testing funcional}{Centro de Ensayos de Software}{}{}{Realizado en la Facultad de Informática de la Universidad de La Plata.}
\cventry{2006}{Supermercados Sur}{Laboratorio de Computación III}{}{}{Implementado en Apache/PHP/JavaScript, es una aplicación que permite administrar un supermercado virtual.}
\cventry{2006}{DES Solar Energy}{Sistemas Expertos}{}{}{Implementado en Smalltalk VisualWorks, y a la par del Ing. Raúl Righini (UNLu), es una aplicación que permite estimar el tamaño de una instalación eléctrica basada en energía solar.}
\cventry{2006}{TSP with Genetic Algorithm v1.0}{Inteligencia Artificial}{}{}{Implementado en Smalltalk VisualWorks, es una aplicación que intenta resolver el "Problema del Viajante" (Travelling Salesman Problem), mediante algorítmos genéticos.}
\cventry{2006}{Tetris v1.2}{Programación Orientada a Objetos}{}{}{Implementado en Smalltalk VisualWorks, es una versión sencilla del tetris.}
\cventry{2005}{Roberto Insausti SA (Pergamino)}{Seminario Profesional}{}{}{Análisis y diseño de posibles soluciones a la empresa Roberto Insausti SA.}
\cventry{2004}{Mosquito v1.4}{Seminario de Actualización I}{}{}{Implementado en Delphi, es una aplicación gráfica que mediante simulación reactiva imita el comportamiento de un mosquito.}
\cventry{2003}{Sistema Bibliotecario v4}{Programación Aplicada}{}{}{Implementado en Perl/Firebird, es una aplicación no gráfica que permite administrar un sistema bibliotecario sencillo.}
\cventry{2002}{Agenda v1.2}{Programación Algortítmica I}{}{}{Implementado en Basic, es una aplicación de agenda gráfica que se ejecuta bajo consola.}


\section{Experiencia Laboral}
% \cventry{year--year}{Job title}{Employer}{City}{}{Description}  % arguments 3 to 6 are optional
\cventry{2018--2020}{Desarrollador de Software Senior}{Freelancer}{Remoto}{}{Colaboré con el desarrollo y mantenimiento de algunos proyectos. Utilizando tecnologias como Flask, Celery, Objective C, SwiftUI, Docker Compose, y algunas otras más.}
\cventry{2020}{Conceptos de Programación para las Biociencias}{Coordinador}{Remoto}{}{Colaboré con el desarrollo y dictado del curso de Master en Bioinformatica y Biología de Sistemas de la UNNOBA.}
\cventry{2016--2017}{Desarrollador de Software para Willdom SA}{Willdom SA}{Remoto}{}{Colaboré con el desarrollo y mantenimiento de la plataforma de ventas on line del proyecto \href{https://www.squaretrade.com/go}{SquareTradeGo} (web y iOS).}
\cventry{2012--2016}{Investigador en \href{http://www.gersol.unlu.edu.ar}{GERSolar}}{Universidad Nacional de Luján}{Luján}{}{Desarrollo de una arquitectura de procesamiento de imágenes satelitales, que tiene el fin  de estimar la radiación solar al nivel del suelo, para todo el área de la República Argentina. Git: https://github.com/gersolar}
\cventry{2010--2012}{Investigador en el \href{http://limiear.github.io}{LIMIE}}{}{Pergamino}{}{Desarrollo de un dispositivo para no videntes.
Lenguajes utilizados: C (embebido en ARM), Ruby, Java (para Android).}
\cventry{2007--2009}{Ayudante de laboratorio en el \href{http://lifia.info.unlp.edu.ar}{LIFIA}}{Universidad Nacional de La Plata}{La Plata}{}{Formé parte del desarrollo de una plataforma de testing para máquinas tragamonedas. Lenguajes utilizados: C++, Python, Lua. Referencia: federico.balaguer@lifia.info.unlp.edu.ar.}
\cventry{2008}{Colaborador en la materia Orientación a Objetos 1}{Universidad Nacional de La Plata}{La Plata}{}{Me desempeñé como colaborador de la materia en la parte práctica. Lenguaje utilizado: Smalltalk.}
\cventry{2006--2008}{Ayudante de segunda en la materia Programación III}{Universidad Nacional de Luján}{Luján}{}{Me desempeñé como ayudante de la materia en la parte práctica. Lenguajes utilizados: Smalltalk, Java. Referencia: alejandro.fernandez@lifia.info.unlp.edu.ar.}

\section{Conocimientos}
\cvcomputer{Modeling}{\textbf{design patterns, refactoring, machine learning}}{DB}{\textbf{firebird, mongodb, mysql, postgresql, sqlite3}}
\cvcomputer{Services}{\textbf{apache, subversion, git, nginx}}{Virtualization}{\textbf{virtualbox, kvm, docker}}
\cvcomputer{Networking}{\textbf{tcp/ip, route, nat, filters, iptables}}{OS}{\textbf{gnu/linux, osx, windows}}
\cvcomputer{Weak typing languages}{\textbf{javascript (cypress, reactjs, angularjs, jquery, knockoutjs, lodash, prototype, protractor, webpack), lisp, lua, octave, perl, php, prolog, python (django, flask, mocker, pytest, sqlalchemy), ruby (rails 5.1), smalltalk (seaside, meteoroid), xml, xsl}}{Strong typing languages}{\textbf{assembly, c/c++ (boost, asio, stl, lpc1343), delphi, java (android), swift (ios), objective-c (ios), \LaTeX, visual basic}}
\cvcomputer{Utilities}{\textbf{gimp, vim, html, css, ssh, makefile, docker}}{}{}

\section{Idiomas}
% \cvlanguage{language 1}{Skill level}{Comment}
\cvlanguage{English}{Middle level}{}
\begin{CJK}{UTF8}{gbsn}
\cvlanguage{汉语}{初始水平}{}
\end{CJK}

\section{Referencias}
% \cventry{MBA Jeff Wang}{wang.jeffc@gmail.com}{}{}{}{}
\cventry{Dr. Gast{\'o}n {\'A}vila}{avila.gas@gmail.com}{}{}{}{}
\cventry{Dr. Ra{\'u}l Righini}{raulrighini@yahoo.com.ar}{}{}{}{}
\cventry{Dr. Alejandro Fern{\'a}ndez}{alejandro.fernandez@lifia.info.unlp.edu.ar}{}{}{}{}
% \cventry{Mg. Gabriel Tolosa}{tolosoft@unlu.edu.ar}{}{}{}{}
\cventry{Mg. Fernando Bordignon}{fernando.bordignon@gmail.com}{}{}{}{}


%% \section{Aficiones}
%% % \cvline{hobby 1}{\small Description}
%% \cvline{Guitarra}{\small Formación elemental tanto de Guitarra Clásica como de Solfeo.}
%% \cvline{Pintura}{\small Asistencia a cursos de pintura durante un periodo de 5 años.}
%% \cvline{Deportes}{\small Aficionado a todos los deportes. Practico paddle, fútbol y baloncesto.}


%\closesection{}                   % needed to renewcommands
\renewcommand{\listitemsymbol}{-} % change the symbol for lists

% section{Extra 1}
% \cvlistitem{Item 1}
% \cvlistitem{Item 2}
% \cvlistitem[+]{Item 3}            % optional other symbol

% section{Extra 2}
% \cvlistdoubleitem[\Neutral]{Item 1}{Item 4}
% \cvlistdoubleitem[\Neutral]{Item 2}{Item 5}
% \cvlistdoubleitem[\Neutral]{Item 3}{}

% Publications from a BibTeX file
% \nocite{*}
% \bibliographystyle{plain}
% \bibliography{publications}       % 'publications' is the name of a BibTeX file

\end{document}


%% end of file `template_en.tex'.
