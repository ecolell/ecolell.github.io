% Chapter 7

\chapter{Conclusión}
% Write in your own chapter title
\label{Capitulo 7}
\lhead{Capitulo 7. \emph{Conclusión}} % Write in your own chapter title to set the page header

Como consecuencia del diseño y el desarrollo realizados, ha quedado demostrada la existencia de una alternativa que permita a una aplicación web desarrollada con continuations utilizar los sensores de un dispositivo.

Teniendo en cuenta los objetivos de esta tesis y las características de las posibles extensiones, y habiendo contemplando a su vez los límites de la idea presentada, a continuación se realizará una comparación con otros trabajos de semejantes características.

Luego se enumerarán los resultados obtenidos a partir del diseño propuesto. Y por último se presentarán algunos de las trabajos pendientes que requieren de un análisis mas profundo.


\section{Trabajos relacionados}
\label{Trabajos Relacionados}

En la introducción a este documento se mencionaron otros trabajos que mantienen relación con alguna de las lineas de investigación desarrolladas. Para distinguir el aporte realizado por este trabajo se procede con la descripción de las principales diferencias entre el diseño presentado y el enfoque presentado por el resto de los trabajos.

Al contemplar el aporte realizado por \emph{Queinnec}\cite{Queinnec01}, que resuelve los inconvenientes relacionados con las transacciones y los estados de una aplicación web mediante la utilización de Continuations, este diseño no hace mas que agregar una extensión que permita el manejo de información contextual a este tipo de aplicaciones.

Luego, siguiendo por la linea de las aplicaciones web se encuentra el trabajo de \emph{Challiol et al.}\cite{Challiol06}, que ha sido desarrollado con MVC. En este trabajo se implementa un Controlador que es el encargado de manejar la información contextual, aunque esta solo consista en información de localización, que puede ser obtenida por múltiples sensores que se encuentran distribuidos en el ambiente.

A diferencia de \emph{Challiol et al.}, en este trabajo se presentan ejemplos en los que se restringe la ubicuidad de los sensores y se determina que la información es siempre proporcionada por el dispositivo móvil del cliente, con el fin que el usuario siempre tenga el control de su privacidad. Además los valores leídos por los sensores nunca son enviados a los servidores, con el fin de mejorar la privacidad y seguridad del usuario.

Por otra parte \emph{Chang et al.}\cite{Chang07} plantea un mecanismo de adaptación web que se enfoca en incorporar la sensibilidad al contexto contemplando principalmente la QoS. Su enfoque consiste en decidir de que lado de la conexión conviene ejecutar una serie de bloques (o componentes). Este tipo de adaptación puede ser realizada solo en el momento que la aplicación web se carga en el cliente, perdiendo la aplicación cualquier posibilidad de realizar nuevas adaptaciones en tiempo de ejecución.

A diferencia de \emph{Chang et al.}, en este trabajo se provee mayor flexibilidad para decidir el momento de aplicar una adaptación. Por otro lado, \emph{Chang et al.} contempla la variante evolutiva al momento de diseñar su modelo.

En el 2009, \emph{Kapitsaki et al.}\cite{Kapitsaki09} define que una forma de permitir la sensibilidad al contexto es mediante la composición de un servicio o aplicación web con un servicio de sensibilidad al contexto, mediante la utilización de otro servicio web que se encarga de la composición. Esto implica que la sensibilidad al contexto queda definida fuera del modelo de negocios.

Parte de estas nociones, proporcionadas por \emph{Kapitsaki et al.}, han sido utilizadas al momento de definir como se relaciona la sensibilidad al contexto con el modelo de negocio. A diferencia de \emph{Kapitsaki et al.}, en este trabajo, la composición entre la sensibilidad al contexto y el modelo de negocio ocurren dentro de los componentes de Seaside.

El resto de los trabajos (\emph{Efstratiou}\cite{Efstratiou04} y \emph{Fortier et al.}\cite{Fortier09}) se encuentran relacionados con la sensibilidad al contexto y proponen el desarrollo de aplicaciones que no corren en navegadores web, por lo que la privacidad del usuario nunca surgió como un tópico a analizar.

El trabajo de \emph{Efstratiou}\cite{Efstratiou04} consiste en desarrollar un procedimiento basado en pólizas que permite administrar de forma coordinada las adaptaciones sugeridas por múltiples aplicaciones. Estas pólizas consisten en disparar \emph{triggers} para llevar a cabo las adaptaciones necesarias.

Aunque en el diseño presentado en el capítulo \ref{Capitulo 4} se utiliza el concepto de triggers de \emph{Efstratiou} y se mencionan posibles extensiones que produzcan una adaptación coordinada, fue necesario adaptar lo propuesto por \emph{Efstratiou} a una arquitectura de aplicaciones web basada en Continuations.

En comparación con la propuesta de \emph{Costanza}\cite{Costanza08}, que presenta el \emph{COP} como un nuevo paradigma, esta librería tiene un alcance mas limitado que es brindar sensibilidad al contexto a aplicaciones web desarrolladas con continuations.

Por último \emph{Fortier et al.}\cite{Fortier09}, plantea una plataforma para tratar con la complejidad de un software sensible al contexto. Esta plataforma está compuesta por 4 partes: el \emph{soporte de sensibilidad}, las \emph{adaptaciones específicas del dominio}, un \emph{adaptador a aplicaciones existentes} y por último el módulo de conexión de las tres partes anteriores compuesto por \emph{abstracciones centrales}.

La principal diferencia con el diseño de \emph{Fortier et al.} consiste en el punto de conexión entre el modelo contextual y el modelo de negocio. \emph{Fortier et al.} considera que el modelo contextual se encuentra directamente relacionado con el modelo de negocios, de esta forma se torna mas difícil que cada interfaz de usuario establezca nuevas formas de comunicar ambos modelos (el contextual y el de negocios).

Para sobrellevar dicha situación, en este documento se presentó un lenguaje que permite definir la relación entre el modelo contextual y el modelo de negocios dentro de una subclase de Component (que se utilizan para definir interfaces en Seaside).


\section{Lecciones aprendidas}

Al concluir este trabajo se ha logrado cumplir con tres los objetivos que lo motivaron: El desarrollo, la explicación mediante patrónes de diseño y la comparación con otras alternativas. 

En primer lugar, se ha proporcionado un mecanismo para que una aplicación web desarrollada con continuations se pueda adaptar al contexto del usuario. Esta adaptación permitirá que aplicaciones web que en la actualidad se encuentran realizadas con continuations puedan a su vez adaptarse al contexto.

Luego, se han explicado las decisiones de diseño mediante la utilización de patrónes de diseño como: \emph{Adapter}, \emph{Composite}, \emph{Publish/Subscriber} y \emph{Builder}.

El \emph{Adapter} ha permitido estandarizar las interfaces de los posibles sensores para que la utilización de sus valores pueda realizarse mediante condiciones preestablecidas.

Luego, el \emph{Composite} ha proporcionado flexibilidad y simplicidad para combinar distintas condiciones, permitiendo a su vez realizar ciertas optimizaciones en el procesamiento.

Posteriormente, el \emph{Publish/Subscriber} ha otorgado una forma de mantener la privacidad del usuario, además de reducir el uso de la red en comparación con otras alternativas basadas en MVC.

El \emph{Builder}, por otra parte, ha simplificado la forma de definir condiciones, permitiendo que el desarrollador siga enfocado el la lógica del negocio, y ahora con la posibilidad de aprovechar la información contextual.

Por último, en la sección \nameref{Trabajos Relacionados} (\ref{Trabajos Relacionados}) se ha realizado una breve mención de las diferencias existentes con otras alternativas.


\section{Trabajo futuro}

Luego de la conclusión del objetivo planteado en este documento, y considerando las distintas alternativas para continuar esta nueva linea de investigación\footnote{Considerando que es necesaria una linea de investigación que se enfoque en analizar a las aplicaciones web basadas en continuations en combinación con la sensibilidad al contexto.}, en un futuro me encontraré analizando el rendimiento, en términos de necesidad computacional y de comunicación de la sensibilidad al contexto, entre las aplicaciones web basadas en continuations y aquellas que utilizan el esquema MVC.

En esta linea, también ha quedado pendiente mejorar los Builders de condiciones tanto simples como complejas de forma tal que utilicen todas las posibilidades de la \emph{lógica difusa} para definir los diferentes entornos.

Por otra parte, dentro del navegador web, es necesario analizar la definición de una API que estandarice la definición de sensores. De esta forma reducirá la necesidad de definir el comportamiento específico de cada \emph{Listener} del lado de \emph{javascript}.

Además, analizaré que posibilidades hay de definir una Ontología que permita la utilización de nuevos sensores sin la necesidad de modificación del navegador web. Esto permitirá agregar nuevos sensores, sin la necesidad de volver a compilar el navegador web. En esta linea, tal vez sea posible detectar de forma automática que sensores se encuentran disponibles en un dispositivo, y ponerlos a disposición para que el usuario pueda utilizarlos.

En otro aspecto, es necesario mejorar e investigar una alternativa que mejore la percepción del usuario en torno a la administración del acceso y la información privada. En esta etapa, será necesario realizar un análisis de usabilidad de la interfaz de configuración y ofrecer una alternativa que sea simple e intuitiva.

Por último, continuando con la administración del acceso, también me enfocaré en analizar la \emph{interacción entre aplicaciones web}, la \emph{administración de permisos mediante Composite}, los \emph{permisos asociativos} y 	posibles \emph{estrategias para extender el Same Origin Policy} (Ver el capítulo \ref{Capitulo 6}, \nameref{Capitulo 6}).