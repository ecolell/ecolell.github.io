\documentclass[a4paper]{article}
\usepackage[spanish]{babel}
\usepackage[utf8]{inputenc}
\usepackage{tabularx}

%\usepackage{doublespace}
%\setstretch{1.2}

\usepackage{ae}
\usepackage[T1]{fontenc}
\usepackage{CV}

\begin{document}

\pagestyle{plain}

%Ueberschrift
\begin{center}
\huge{\textsc{Curriculum Vitae}}
\vspace{\baselineskip}

\Large{\textsc{Eloy Adonis Colell}}
\vspace{\baselineskip}

\small{\textsc{Licenciado en Sistemas de Información}}
\end{center}
\vspace{1.5\baselineskip}

\section{Dirección}
\begin{flushleft}
San Nicolás 2310 \\
Pergamino \\
Buenos Aires \\
Argentina \\
Código Postal: B2700LCR \\
Telefono: +54 (02477) 1553-6539 \\
Email: eloy.colell.jobs@gmail.com \\
Homepage: ecolell.github.io\\
\end{flushleft}

\section{Datos Personales}
\begin{flushleft}
Sexo: Masculino \\
Estado Civil: Soltero \\
Edad: 31 años \\
Fecha de nacimiento: 8 de Agosto de 1983 \\
Lugar de nacimiento: Pergamino, Argentina \\
DNI: 30.344.242 \\
\end{flushleft}

\section{Educación}
\begin{CV}
\item[1989--1996] EGB-Escuela Nº 62 - Pergamino
\item[1997--1998] EGB-Escuela Nº 54 - Rancagua
\item[1999--2001] Instituto Comercial Rancagua
  \begin{itemize}
  \item Bachiller Polimodal (modalidad Ciencias Naturales)
  	\begin{description}
  	\item[Talleres extracurriculares: ] Periodismo, Estadística, Contabilidad
  	\item[Intercambios estudiantiles: ] Colegio Santa María Goretti (Rancagua de Chile)
  	\end{description}
  \end{itemize}
\item[2002--2013] Universidad Nacional de Luján
  \begin{itemize}
  \item Analísta de Sistemas de Información
  \item Licenciado en Sistemas de Información
  \end{itemize}
\item[2014--    ] Universidad Nacional del Noroeste de la Provincia de Buenos Aires
  \begin{itemize}
  \item Master en Bioinformática y Biología de Sistemas
  \end{itemize}
\end{CV}

\section{Conocimientos tecnológicos}
\begin{CV}
\item[Modeling] design patterns, refactoring
\item[DB] firebird, mysql, postgresql, sqlite3
\item[Services] apache, subversion, git
\item[Virtualization] virtualbox, kvm
\item[Networking] tcp/ip, route, nat, filters
\item[OS] gnu linux, osx, windows
\item[Scripting] javascript (prototype, jquery), lisp, lua, octave, perl, php, prolog, python (django), ruby (rails, wee), smalltalk (seaside, meteoroid), xml, xsl
\item[Compiled] assembly, c/c++, (boost, asio, stl, lpc1343), delphi, java (android), \LaTeX, visual basic
\item[Utilities] gimp, vim, html, css, ssh, makefile
\end{CV}

\section{Otros conocimientos}
\begin{itemize}
\item Inglés (nivel intermedio)
\end{itemize}

\section{Experiencia Académica}
\begin{description}
\item [ Preparación y evaluación de proyectos ] Adaptabilidad al contexto en aplicaciones web desarrolladas con continuations.\\
Se crea una librería para el desarrollo de aplicaciones web que facilita la utilización de los sensores provistos por un navegador web. Este trabajo mezcla 2 lineas de investigación: la sensibilidad al contexto y las aplicaciones web desarrolladas con continuations.
\item [ Especialización en Universidad de Jaén ] Estimación radiación solar\\
Utilización de imágenes de uno de los satélites Meteosat Segunda Generación para estimar la radiación solar en Andalucía, España.
\item [ Seminario de Coaching y Liderazgo ] Dictado por Lic. Fabiola Robin Marquez\\
Se introducen los tipos de liderazgos, la motivación, la influencia, el conflicto/mediación/resolución, la toma de decisiones y el coaching.
\item [ Seminario de Actualización 2 ] Conceptos Básicos de Geoestadística\\
Libro introducción a los conceptos básicos de la Geoestadística.
\item [ Fundamentos del testing y técnicas de testing funcional ] CES\\
Organizado por el Centro de Esnayos de Software (CES) en la Facultad de Informática de la Universidad de La Plata.
\item [ Laboratorio de Computación 3 ] Supermercados Sur\\
Implementado en Apache/PHP/JavaScript, es una aplicación que permite administrar un supermercado virtual.
\item [ Sistemas Expertos ] DES Solar Energy\\
Implementado en Smalltalk VisualWorks, y a la par del Ing. Raúl Righini (UNLu),  es una aplicación que permite estimar el tamaño de una instalación electrica basada en energía solar.
\item [ Inteligencia Artificial ] TSProblem with Genetic Algorithm v1.0\\
Implementado en Smalltalk VisualWorks, es una aplicación que intenta resolver el ``Problema del Viajante'' (Travelling Salesman Problem), mediante algoritmos genéticos.
\item [ Programación Orientada a Objetos ] Tetris v1.2\\
Implementado en Smalltalk VisualWorks, es una versión sencilla del tetris.
\item [ Seminario Profesional ] Roberto Insausti SA (Pergamino)
Análisis de la empresa Roberto Insausti SA, y diseño de posibles soluciones a los problemas detectados.
\item [ Seminario de Actualización ] Mosquito v1.4\\
Implementado en Delphi, es una aplicación gráfica que mediante simulación reactiva imita el comportamiento de un mosquito.
\item [ Programación Aplicada ] Sistema Bibliotecario v4\\
Implementado en Perl/Firebird, es una aplicación no gráfica que permite administrar un sistema bibliotecario sencillo.
\item [ Programación Algorítmica 1 ] Agenda v1.2\\
Implementado en Basic, es una aplicación de agenda gráfica que corre bajo consola.
\end{description}

\section{Experiencia Laboral}
\begin{description}
\item[09/2012--presente] Investigador en GERSolar
	\begin{description}
	\item[Lugar: ] Investigador en el Grupo de Estudio de la Radiación Solar (GERSolar) de la Universidad Nacional de Luján.
	\item[Web: ] http://www.gersol.unlu.edu.ar
	\item[Reseña: ]  Desarrollo de una arquitectura de procesamiento de imágenes satelitales, que tiene el fin  de estimar la radiación solar al nivel del suelo, para todo el área de la República Argentina.
	\item[Git: ] https://github.com/gersolar
	\end{description}
\item[08/2010--12/2012] Investigador
	\begin{description}
	\item[Lugar: ] Laboratorio de Investigación en Modelos Informáticos y Electrónicos (LIMIE).
	\item[Web: ] http://limiear.github.io
	\item[Reseña: ] Desarrollo de un dispositivo para no videntes.
	\end{description}
\item[12/2007--08/2009] Asistente de Investigación
	\begin{description}
	\item[Lugar: ] Laboratorio de Investigación y Formación en Informática Avanzada (LIFIA). Facultad de Informática de la Universidad Nacional de La Plata.
	\item[Web: ] http://www.lifia.info.unlp.edu.ar
	\item[Reseña: ] Refactorizar un software comercial extrayendo un subsistema a un proceso separado [http://catalogo.info.unlp.edu.ar/meran/getDocument.pl?id=527].
	\end{description}
\item[08/2006--08/2008] Ayudante de segunda
	\begin{description}
	\item[Cátedra: ] Programación Orientada a Objetos/Programación 3.
	\item[Lugar: ] Universidad Nacional de Luján (UNLu)
	\item[Titular: ] Alejandro Fernández (alejandro.casco.fernandez@gmail.com)
    \item[Web: ] http://www.unlu.edu.ar
	\end{description}
\item[06/2008--12/2008] Colaborador
	\begin{description}
	\item[Cátedra: ] Orientación a Objetos 1.
	\item[Lugar: ] Universidad Nacional de La Plata (UNLP)
	\item[Titulares: ] Ms. Roxana Giandini, Dra. Alicia Díaz
    \item[Web: ] http://www.info.unlp.edu.ar/index.php
	\end{description}
\end{description}

\section{Referencias}
\noindent Estas personas dan fé de mi calidad profesional y mi caracter:
\begin{table}[h]
\begin{tabular}{@{}lll@{}}
\textbf{Dr. Alejandro Fernández} \\ alejandro.fernandez@lifia.info.unlp.edu.ar
\end{tabular}
\begin{tabular}{@{}lll@{}}
\textbf{Dr. Federico Balaguer}\\ federico.balaguer@lifia.info.unlp.edu.ar
\end{tabular}
\begin{tabular}{@{}lll@{}}
\textbf{Dr. Raúl Righini}\\ raulrighini@yahoo.com.ar
\end{tabular}
\end{table}


\vspace{2\baselineskip}
\noindent Pergamino, \today



\end{document}

%Tabellen
\begin{table}[htbp] \centering%
\begin{tabular}{lll}\hline\hline
1 & 2 & 3 \\ \hline
1 & \multicolumn{2}{c}{2} \\
\hline
\end{tabular}
\caption{Titel\label{Tabelle: Label}}
\end{table}
