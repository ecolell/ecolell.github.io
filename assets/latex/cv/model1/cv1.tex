\documentclass[a4paper]{article}
\usepackage{CJK}
\usepackage[spanish]{babel}
\usepackage[utf8]{inputenc}
\usepackage{tabularx}


%\usepackage{doublespace}
%\setstretch{1.2}

\usepackage{ae}
\usepackage[T1]{fontenc}
\usepackage{CV}

\begin{document}


\pagestyle{plain}

%Ueberschrift
\begin{center}
\huge{\textsc{Curriculum Vitae}}
\vspace{\baselineskip}

\Large{\textsc{Eloy Adonis Colell}}
\vspace{\baselineskip}

\small{\textsc{Licenciado en Sistemas de Informaci{\'o}n}}
\end{center}
\vspace{1.5\baselineskip}

\section{Contacto}
\begin{flushleft}
San Nicol{\'a}s 2310 \\
Pergamino \\
Buenos Aires \\
Argentina \\
C{\'o}digo Postal: B2700LCR \\
Telefono: +54 (02477) 1553-6539 \\
Email: eloy.colell.jobs@gmail.com \\
Homepage: ecolell.github.io\\
\end{flushleft}

\section{Datos Personales}
\begin{flushleft}
Sexo: Masculino \\
Estado Civil: Soltero \\
Edad: 31 a\~{n}os \\
Fecha de nacimiento: 8 de Agosto de 1983 \\
Lugar de nacimiento: Pergamino, Argentina \\
DNI: 30.344.242 \\
\end{flushleft}

\section{Educaci{\'o}n}
\begin{CV}
\item[1989--1996] EGB-Escuela Nº 62 - Pergamino
\item[1997--1998] EGB-Escuela Nº 54 - Rancagua
\item[1999--2001] Instituto Comercial Rancagua
  \begin{itemize}
  \item Bachiller Polimodal (modalidad Ciencias Naturales)
  	\begin{description}
  	\item[Talleres extracurriculares: ] Periodismo, Estad{\'i}stica, Contabilidad
  	\item[Intercambios estudiantiles: ] Colegio Santa Mar{\'i}a Goretti (Rancagua de Chile)
  	\end{description}
  \end{itemize}
\item[2002--2013] Universidad Nacional de Luj{\'a}n
  \begin{itemize}
  \item Anal{\'i}sta de Sistemas de Informaci{\'o}n
  \item Licenciado en Sistemas de Informaci{\'o}n
  \end{itemize}
\item[2014--    ] Universidad Nacional del Noroeste de la Provincia de Buenos Aires
  \begin{itemize}
  \item Master en Bioinform{\'a}tica y Biolog{\'i}a de Sistemas \\ (364 hs cursadas / 576 hs )
  \end{itemize}
\end{CV}

\section{Conocimientos tecnol{\'o}gicos}
\begin{CV}
\item[Modeling] design patterns, refactoring
\item[DB] firebird, mysql, postgresql, sqlite3
\item[Services] apache, subversion, git
\item[Virtualization] virtualbox, kvm
\item[Networking] tcp/ip, route, nat, filters
\item[OS] gnu linux, osx, windows
\item[Scripting] javascript (prototype, jquery), lisp, lua, octave, perl, php, prolog, python (django), ruby (rails, wee), smalltalk (seaside, meteoroid), xml, xsl
\item[Compiled] assembly, c/c++, (boost, asio, stl, lpc1343), delphi, java (android), \LaTeX, visual basic
\item[Utilities] gimp, vim, html, css, ssh, makefile
\end{CV}

\begin{CJK}{UTF8}{gbsn}
\section{Otros conocimientos}
\begin{itemize}
\item Ingl{\'e}s (Nivel interm{\'e}dio)
\item 汉语 (初始水平)
\end{itemize}
\end{CJK}

\section{Experiencia Laboral}
\begin{description}
\item[09/2012--presente] Investigador en GERSolar
	\begin{description}
	\item[Lugar: ] Investigador en el Grupo de Estudio de la Radiaci{\'o}n Solar (GERSolar) de la Universidad Nacional de Luj{\'a}n.
	\item[Web: ] http://www.gersol.unlu.edu.ar
	\item[Rese\~na: ]  Desarrollo de una arquitectura de procesamiento de im{\'a}genes satelitales, que tiene el fin  de estimar la radiaci{\'o}n solar al nivel del suelo, para todo el {\'a}rea de la Rep{\'u}blica Argentina.
	\item[Git: ] https://github.com/gersolar
	\end{description}
\item[08/2010--12/2012] Investigador
	\begin{description}
	\item[Lugar: ] Laboratorio de Investigaci{\'o}n en Modelos Inform{\'a}ticos y Electr{\'o}nicos (LIMIE).
	\item[Web: ] http://limiear.github.io
	\item[Rese\~na: ] Desarrollo de un dispositivo para no videntes.
	\end{description}
\item[12/2007--08/2009] Asistente de Investigaci{\'o}n
	\begin{description}
	\item[Lugar: ] Laboratorio de Investigaci{\'o}n y Formaci{\'o}n en Inform{\'a}tica Avanzada (LIFIA). Facultad de Inform{\'a}tica de la Universidad Nacional de La Plata.
	\item[Web: ] http://www.lifia.info.unlp.edu.ar
	\item[Rese\~na: ] Refactorizar un software comercial extrayendo un subsistema a un proceso separado [http://catalogo.info.unlp.edu.ar/meran/getDocument.pl?id=527].
	\end{description}
\item[08/2006--08/2008] Ayudante de segunda
	\begin{description}
	\item[C{\'a}tedra: ] Programaci{\'o}n Orientada a Objetos/Programaci{\'o}n 3.
	\item[Lugar: ] Universidad Nacional de Luj{\'a}n (UNLu)
	\item[Titular: ] Alejandro Fern{\'a}ndez (alejandro.casco.fernandez@gmail.com)
    \item[Web: ] http://www.unlu.edu.ar
	\end{description}
\item[06/2008--12/2008] Colaborador
	\begin{description}
	\item[C{\'a}tedra: ] Orientaci{\'o}n a Objetos 1.
	\item[Lugar: ] Universidad Nacional de La Plata (UNLP)
	\item[Titulares: ] Ms. Roxana Giandini, Dra. Alicia D{\'i}az
    \item[Web: ] http://www.info.unlp.edu.ar/index.php
	\end{description}
\end{description}

\section{Experiencia Acad{\'e}mica}
\begin{description}
\item [ Preparaci{\'o}n y evaluaci{\'o}n de proyectos ] Adaptabilidad al contexto en aplicaciones web desarrolladas con continuations.\\
Se crea una librer{\'i}a para el desarrollo de aplicaciones web que facilita la utilizaci{\'o}n de los sensores provistos por un navegador web. Este trabajo mezcla 2 lineas de investigaci{\'o}n: la sensibilidad al contexto y las aplicaciones web desarrolladas con continuations.
\item [ Especializaci{\'o}n en Universidad de Ja{\'e}n ] Estimaci{\'o}n radiaci{\'o}n solar\\
Utilizaci{\'o}n de im{\'a}genes de uno de los sat{\'e}lites Meteosat Segunda Generaci{\'o}n para estimar la radiaci{\'o}n solar en Andaluc{\'i}a, Espa\~{n}a.
\item [ Seminario de Coaching y Liderazgo ] Dictado por Lic. Fabiola Robin Marquez\\
Se introducen los tipos de liderazgos, la motivaci{\'o}n, la influencia, el conflicto/mediaci{\'o}n/resoluci{\'o}n, la toma de decisiones y el coaching.
\item [ Seminario de Actualizaci{\'o}n 2 ] Conceptos B{\'a}sicos de Geoestad{\'i}stica\\
Libro introducci{\'o}n a los conceptos b{\'a}sicos de la Geoestad{\'i}stica.
\item [ Fundamentos del testing y t{\'e}cnicas de testing funcional ] CES\\
Organizado por el Centro de Esnayos de Software (CES) en la Facultad de Inform{\'a}tica de la Universidad de La Plata.
\item [ Laboratorio de Computaci{\'o}n 3 ] Supermercados Sur\\
Implementado en Apache/PHP/JavaScript, es una aplicaci{\'o}n que permite administrar un supermercado virtual.
\item [ Sistemas Expertos ] DES Solar Energy\\
Implementado en Smalltalk VisualWorks, y a la par del Ing. Ra{\'u}l Righini (UNLu),  es una aplicaci{\'o}n que permite estimar el tama\~{n}o de una instalaci{\'o}n electrica basada en energ{\'i}a solar.
\item [ Inteligencia Artificial ] TSProblem with Genetic Algorithm v1.0\\
Implementado en Smalltalk VisualWorks, es una aplicaci{\'o}n que intenta resolver el ``Problema del Viajante'' (Travelling Salesman Problem), mediante algoritmos gen{\'e}ticos.
\item [ Programaci{\'o}n Orientada a Objetos ] Tetris v1.2\\
Implementado en Smalltalk VisualWorks, es una versi{\'o}n sencilla del tetris.
\item [ Seminario Profesional ] Roberto Insausti SA (Pergamino)
An{\'a}lisis de la empresa Roberto Insausti SA, y dise\~{n}o de posibles soluciones a los problemas detectados.
\item [ Seminario de Actualizaci{\'o}n ] Mosquito v1.4\\
Implementado en Delphi, es una aplicaci{\'o}n gr{\'a}fica que mediante simulaci{\'o}n reactiva imita el comportamiento de un mosquito.
\item [ Programaci{\'o}n Aplicada ] Sistema Bibliotecario v4\\
Implementado en Perl/Firebird, es una aplicaci{\'o}n no gr{\'a}fica que permite administrar un sistema bibliotecario sencillo.
\item [ Programaci{\'o}n Algor{\'i}tmica 1 ] Agenda v1.2\\
Implementado en Basic, es una aplicaci{\'o}n de agenda gr{\'a}fica que corre bajo consola.
\end{description}

\section{Referencias}
\noindent Estas personas dan f{\'e} de mi calidad profesional y mi caracter:
\begin{table}[h]
\begin{tabular}{@{}lll@{}}
\textbf{Dr. Alejandro Fern{\'a}ndez} \\ alejandro.fernandez@lifia.info.unlp.edu.ar
\end{tabular}
\begin{tabular}{@{}lll@{}}
\textbf{Dr. Federico Balaguer}\\ federico.balaguer@lifia.info.unlp.edu.ar
\end{tabular}
\begin{tabular}{@{}lll@{}}
\textbf{Dr. Ra{\'u}l Righini}\\ raulrighini@yahoo.com.ar
\end{tabular}
\end{table}


\vspace{2\baselineskip}
\noindent Pergamino, \today


\end{document}

%Tabellen
\begin{table}[htbp] \centering%
\begin{tabular}{lll}\hline\hline
1 & 2 & 3 \\ \hline
1 & \multicolumn{2}{c}{2} \\
\hline
\end{tabular}
\caption{Titel\label{Tabelle: Label}}
\end{table}
